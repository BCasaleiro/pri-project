\documentclass[a4paper,titlepage,11pt]{article}

\usepackage[top=4cm, bottom=3.54cm, left=3cm, right=3cm]{geometry}

\usepackage[portuguese]{babel}
\usepackage[utf8]{inputenc}
\usepackage{hyperref}

\begin{document}

\begin{titlepage}
  \begin{center}
    {\scshape \huge Projecto \par}
    \vspace{1cm}

    {\scshape \LARGE Parte 2 \par}
    \vspace{1.5cm}

    {\scshape \Large Processamento e Recuperação \\ de Informação \par}
    \vfill

    {\itshape \Large Grupo 24 \par}
    \vfill

    \begin{tabular}{l l}
      Bernardo Casaleiro & 87827\\
      Rui Oliveira & 70604\\
    \end{tabular}
    \vfill

    {\large \today\par}
  \end{center}
\end{titlepage}

\section{An approach based on graph ranking}
No primeiro exercício foi implementado o algoritmo pedido de forma muito simples usando apenas a
biblioteca \href{http://nltk.org/}{nltk} por forma a fazer download das stopwords.

O algoritmo começa por ler o ficheiro desejado, gerando em seguida um grafo onde cada nó corresponde a
um n-grama com 1 \textless= n \textless= 3, ignorando as stopwords,  e as arestas entre estes representam
a co-ocorrência na mesma frase, sendo assim não direccionados.

O grafo gerado é então usado para calcular o ranking de acordo com o algoritmo page rank num processo
iterativo com 50 iterações.

Por fim foi então aplicado o algoritmo ao ficheiro '\textit{document.txt}' sendo os resultados:

\begin{table}[ht]
  \centering
  \begin{tabular}{ll}
    gram dry gland & 2.4984416512e-07 \\
    rat parotid gland & 2.49844199144e-07 \\
    milliliters perfused & 2.498442989e-07 \\
    quantity saliva & 2.498442989e-07 \\
    perfused gram dry & 2.50039652151e-07
  \end{tabular}
  \caption{Resultados}
\end{table}

\section{Improving the graph-ranking method}
O exercicio 2 consistiu num upgrade ao exercicio 1, atribuindo o valor Prior a cada nó e atribuir um peso a
cada aresta.

Foram assim implementados como possiveis Prior o \textbf{número de ocorrências}, \textbf{posição na texto} e
\textbf{tf-idf score} do n-grama, bem como para peso das arestas o \textbf{número de ocorrências} da co-ocorrência
de dois n-gramas.

Após vários testes efetuados foi claro para nós que a variação com melhores resultados foi usando \textbf{tf-idf} como Prior
e o \textbf{número de ocorrências} da co-ocorrência de dois n-gramas como peso das arestas. Sendo com estes valores calculados
os resultados apresentados. Foi também necessário reescrever o código por forma a melhorar a performance, reduzindo significativamente
o tempo de execução. No entanto, mesmo assim, de modo a aplicar o algoritmo ao dataset apresentado foram necessárias 67 horas.

Aplicando ambos os exercicios ao dataset escolhido, que se encontra em anexo, obteve-se uma precisão média de \textbf{0.0066667} para o
exercicio 1 e \textbf{0,1416667} para o exercicio 2. Provando assim uma melhoria significativa nos resultados.

Encontram-se em anexo os ficheiros '\textit{results-1.txt}' e '\textit{results-2.txt}' com o output resultante da análise do dataset
usando, respectivamente, o exercicio 1 e 2, pois apesar da precisão média ser um indicador da eficácia destes algoritmos, a nosso ver
não é um bom identificador pois existem casos em que os algoritmos implementados escolhem bons ngramas ou, por exemplo, apenas uma
palavra do n-grama sugerido nos ficheiros key.

\newpage

\section{A supervised learning-to-rank approach}
O algoritmo que optamos por implementar para realizar este exercício foi o Perceptrão.

De modo a comparar a solução desenvolvida neste exercício com a desenvolvida no exercicio 2
foi usado como  ficheiro de teste o mesmo dataset que no exercicio 2, usando outros dos datasets
\footnote{ \href{https://github.com/zelandiya/keyword-extraction-datasets}{https://github.com/zelandiya/keyword-extraction-datasets} }
para treinar o algoritmo.

Foram consideradas todos os n-gramas com 1 \textless= n \textless= 3 de cada documento após a remoção das stopwords.
Sendo as propriedades consideradas para a classificação de cada n-grama:

\begin{itemize}
  \item Se todos os termos do n-grama se encontram entre as primeiras 250 palavras do documento.
  Consideramos que um abstract tem cerca de 200 palavras, portanto com o limite de 250 esperamos na
  maior parte dos casos abranger o titulo, abstract e as primeiras frases do documento.
  \item O valor do n-grama de acordo com a função de classificação BM25 recorrendo a código desenvolvido
  na primeira parte do projecto.
  \item O quão o n-grama se aproxima duma frase, recorrendo à função pos\_tag() da biblioteca nltk e comparando
  as classes gramaticais obtidas com a expressão regular
  \{(\textless{NN}.*\textgreater+ \textless{IN}\textgreater)? \textless{JJ}\textgreater* \textless{NN}.*\textgreater+\}
  que pretende representar uma frase.
  \item O valor da centralidade do n-grama no grafo gerado pelo respectivo documento.
\end{itemize}

Após o treino estar completo e obtidos os pesos das propriedades e um adicional correspondente ao bias,
obtém-se o valor de cada n-grama do documento a classificar multiplicando cada propriedade pelo peso
respectivo. De seguida ordenam-se os n-gramas utilizando este valor e devolve-se os 5 com melhor classificação.

Finalmente, com base nas keywords já conhecidas, calcula-se a precisão media do método para se comparar
com o valor obtido no exercício anterior.

\section{A practical application}
Para o exercicio final foi utilizado a implementação do exercicio 2, mas utilizando a posição do n-grama na frase
como valor Prior ao invés do tf-idf score desse mesmo n-grama.

O programa começa por ler o ficheiro XML/RSS e dar parse de todos os items e em seguida aplicar o algoritmo implementado
no exercicio 2, sendo o resultado impresso no ficheiro '\textit{index.html}'.

Por forma a tornar visualização dos dados mais apelativa foi utilizada a framework \href{http://getbootstrap.com/}{Bootstrap}
e as tabelas fornecidas pelo mesmo.

Encontra-se em anexo um ficheiro '\textit{index.html}' gerado pela nossa implementação a titulo de exemplo.

\end{document}
